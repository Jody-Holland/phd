% Options for packages loaded elsewhere
% Options for packages loaded elsewhere
\PassOptionsToPackage{unicode}{hyperref}
\PassOptionsToPackage{hyphens}{url}
\PassOptionsToPackage{dvipsnames,svgnames,x11names}{xcolor}
%
\documentclass[
  11pt,
  a4paper,
]{article}
\usepackage{xcolor}
\usepackage[top=2.5cm,bottom=2.5cm,left=2.5cm,right=2.5cm]{geometry}
\usepackage{amsmath,amssymb}
\setcounter{secnumdepth}{-\maxdimen} % remove section numbering
\usepackage{iftex}
\ifPDFTeX
  \usepackage[T1]{fontenc}
  \usepackage[utf8]{inputenc}
  \usepackage{textcomp} % provide euro and other symbols
\else % if luatex or xetex
  \usepackage{unicode-math} % this also loads fontspec
  \defaultfontfeatures{Scale=MatchLowercase}
  \defaultfontfeatures[\rmfamily]{Ligatures=TeX,Scale=1}
\fi
\usepackage{lmodern}
\ifPDFTeX\else
  % xetex/luatex font selection
\fi
% Use upquote if available, for straight quotes in verbatim environments
\IfFileExists{upquote.sty}{\usepackage{upquote}}{}
\IfFileExists{microtype.sty}{% use microtype if available
  \usepackage[]{microtype}
  \UseMicrotypeSet[protrusion]{basicmath} % disable protrusion for tt fonts
}{}
\usepackage{setspace}
\makeatletter
\@ifundefined{KOMAClassName}{% if non-KOMA class
  \IfFileExists{parskip.sty}{%
    \usepackage{parskip}
  }{% else
    \setlength{\parindent}{0pt}
    \setlength{\parskip}{6pt plus 2pt minus 1pt}}
}{% if KOMA class
  \KOMAoptions{parskip=half}}
\makeatother
% Make \paragraph and \subparagraph free-standing
\makeatletter
\ifx\paragraph\undefined\else
  \let\oldparagraph\paragraph
  \renewcommand{\paragraph}{
    \@ifstar
      \xxxParagraphStar
      \xxxParagraphNoStar
  }
  \newcommand{\xxxParagraphStar}[1]{\oldparagraph*{#1}\mbox{}}
  \newcommand{\xxxParagraphNoStar}[1]{\oldparagraph{#1}\mbox{}}
\fi
\ifx\subparagraph\undefined\else
  \let\oldsubparagraph\subparagraph
  \renewcommand{\subparagraph}{
    \@ifstar
      \xxxSubParagraphStar
      \xxxSubParagraphNoStar
  }
  \newcommand{\xxxSubParagraphStar}[1]{\oldsubparagraph*{#1}\mbox{}}
  \newcommand{\xxxSubParagraphNoStar}[1]{\oldsubparagraph{#1}\mbox{}}
\fi
\makeatother


\usepackage{longtable,booktabs,array}
\usepackage{calc} % for calculating minipage widths
% Correct order of tables after \paragraph or \subparagraph
\usepackage{etoolbox}
\makeatletter
\patchcmd\longtable{\par}{\if@noskipsec\mbox{}\fi\par}{}{}
\makeatother
% Allow footnotes in longtable head/foot
\IfFileExists{footnotehyper.sty}{\usepackage{footnotehyper}}{\usepackage{footnote}}
\makesavenoteenv{longtable}
\usepackage{graphicx}
\makeatletter
\newsavebox\pandoc@box
\newcommand*\pandocbounded[1]{% scales image to fit in text height/width
  \sbox\pandoc@box{#1}%
  \Gscale@div\@tempa{\textheight}{\dimexpr\ht\pandoc@box+\dp\pandoc@box\relax}%
  \Gscale@div\@tempb{\linewidth}{\wd\pandoc@box}%
  \ifdim\@tempb\p@<\@tempa\p@\let\@tempa\@tempb\fi% select the smaller of both
  \ifdim\@tempa\p@<\p@\scalebox{\@tempa}{\usebox\pandoc@box}%
  \else\usebox{\pandoc@box}%
  \fi%
}
% Set default figure placement to htbp
\def\fps@figure{htbp}
\makeatother





\setlength{\emergencystretch}{3em} % prevent overfull lines

\providecommand{\tightlist}{%
  \setlength{\itemsep}{0pt}\setlength{\parskip}{0pt}}



 


\usepackage{booktabs}
\usepackage{longtable}
\usepackage{array}
\usepackage{multirow}
\usepackage{wrapfig}
\usepackage{float}
\usepackage{colortbl}
\usepackage{pdflscape}
\usepackage{tabu}
\usepackage{threeparttable}
\usepackage{threeparttablex}
\usepackage[normalem]{ulem}
\usepackage{makecell}
\usepackage{xcolor}
\makeatletter
\@ifpackageloaded{caption}{}{\usepackage{caption}}
\AtBeginDocument{%
\ifdefined\contentsname
  \renewcommand*\contentsname{Table of contents}
\else
  \newcommand\contentsname{Table of contents}
\fi
\ifdefined\listfigurename
  \renewcommand*\listfigurename{List of Figures}
\else
  \newcommand\listfigurename{List of Figures}
\fi
\ifdefined\listtablename
  \renewcommand*\listtablename{List of Tables}
\else
  \newcommand\listtablename{List of Tables}
\fi
\ifdefined\figurename
  \renewcommand*\figurename{Figure}
\else
  \newcommand\figurename{Figure}
\fi
\ifdefined\tablename
  \renewcommand*\tablename{Table}
\else
  \newcommand\tablename{Table}
\fi
}
\@ifpackageloaded{float}{}{\usepackage{float}}
\floatstyle{ruled}
\@ifundefined{c@chapter}{\newfloat{codelisting}{h}{lop}}{\newfloat{codelisting}{h}{lop}[chapter]}
\floatname{codelisting}{Listing}
\newcommand*\listoflistings{\listof{codelisting}{List of Listings}}
\makeatother
\makeatletter
\makeatother
\makeatletter
\@ifpackageloaded{caption}{}{\usepackage{caption}}
\@ifpackageloaded{subcaption}{}{\usepackage{subcaption}}
\makeatother
\usepackage{bookmark}
\IfFileExists{xurl.sty}{\usepackage{xurl}}{} % add URL line breaks if available
\urlstyle{same}
\hypersetup{
  pdftitle={Week 3 Progress Report},
  pdfauthor={Jody Holland},
  colorlinks=true,
  linkcolor={blue},
  filecolor={Maroon},
  citecolor={Blue},
  urlcolor={Blue},
  pdfcreator={LaTeX via pandoc}}


\title{Week 3 Progress Report}
\usepackage{etoolbox}
\makeatletter
\providecommand{\subtitle}[1]{% add subtitle to \maketitle
  \apptocmd{\@title}{\par {\large #1 \par}}{}{}
}
\makeatother
\subtitle{Revised Reports}
\author{Jody Holland}
\date{2025-11-06}
\begin{document}
\maketitle


\setstretch{1.25}
\textbf{Previous Meeting Recap:}

\begin{itemize}
\item
  Agreement to revise the structure of the PPP reports.
\item
  A bullet-point draft of the feasibility report.
\item
  Causal pathways for the local leakage chapter were established, with
  initial considerations on the data and methods required to test them.
\end{itemize}

\textbf{Progress}

Following our discussion, I have refined the structure for the first
chapter on local leakage. The approach builds upon the work of Ford et
al.~(2020) and Fuller et al.~(2020) but introduces a crucial difference
in research design. Moving beyond solely a comparison of rates inside
and outside a Protected Area (PA), I am designing a deductive approach.
This constructs causal pathways between the context of the PA and its
potential local impacts on other habitats within the landscape. These
lead to testable hypotheses which can be examined using comparative
methods between landscapes with different features.

To this end, I have developed a foundational pathway that links
variables - such as the factor mobility of economic activities and the
landscape's connectedness to other markets - to the propensity for
production displacement either within or outside of the landscape.

\includegraphics[width=4.40625in,height=\textheight,keepaspectratio]{images/clipboard-3756533519.png}

This mix of variables creates a variety of causal pathways. Building on
our last discussion, I have outlined several of these:

\begin{longtable}[]{@{}
  >{\raggedright\arraybackslash}p{(\linewidth - 6\tabcolsep) * \real{0.2500}}
  >{\raggedright\arraybackslash}p{(\linewidth - 6\tabcolsep) * \real{0.2222}}
  >{\raggedright\arraybackslash}p{(\linewidth - 6\tabcolsep) * \real{0.3056}}
  >{\raggedright\arraybackslash}p{(\linewidth - 6\tabcolsep) * \real{0.2222}}@{}}
\toprule\noalign{}
\endhead
\bottomrule\noalign{}
\endlastfoot
\textbf{Context / Dominant Land Use} & \textbf{Key Driver / Pressure} &
\textbf{Mechanism / Spatial Effect} & \textbf{Outcome / Signal} \\
Area dominated by subsistence farming (Low factor mobility) & Population
pressure mandates expansion of local food production. & Areas in
proximity to the PA experience more complete land clearance (fewer
untouched habitat patches). & Strong local leakage signal. \\
Area dominated by mobile commodity/plantation agriculture (e.g., palm
oil, soy) & Market pressures encourage expansion in many locations, but
PA stifles local investment. & Production pressure is displaced to other
landscapes. & Lower local leakage signal; landscape exhibits potential
``blockage'' traits. \\
PA with spatially heterogeneous additionality (some areas more protected
than others) in a region of low factor mobility. & Underlying pressure
for land conversion remains. & Areas in proximity to sites of high
additionality within the PA experience more complete land clearance. &
Spatially specific local leakage signal. \\
Area dominated by landscape-specific commodities (e.g., coffee, vanilla)
(Low factor mobility due to product specificity) & Global Market
pressures encourage expansion of local production. & Due to product
specificity, pressure cannot be easily absorbed by other landscapes.
Areas near the PA experience more complete land clearance. & Strong
local leakage signal. \\
\end{longtable}

To test these hypothesised pathways, data on the local economies of
landscapes with PAs is needed. I plan to use models like TESSERA to
estimate land use proportions near projects. I have begun experimenting
with downloading the embeddings and will develop a streamlined pipeline
for producing land use classification maps.

\pandocbounded{\includegraphics[keepaspectratio]{images/clipboard-263670576.png}}

Additionally, I have developed a more computationally efficient version
of the find\_potential\_matches script used in PACT. The original was
too intensive for the large-scale pixel matching required for this
chapter, especially on a shared resource like Sherwood. A flowchart
outlining the new approach can be seen \href{images/build_m.png}{here}.

Finally, the main draft of my feasibility report is largely complete and
can be accessed
\href{https://universityofcambridgecloud-my.sharepoint.com/:w:/g/personal/jh2589_cam_ac_uk/EQEyQXRg9ctDhhbhwKabcOEBm7NT-saiZQnxcg4mlevDhw?e=X0fJxz}{here}.
I would appreciate any feedback you may have on its content and
structure.

\textbf{Problems}

TESSERA Embeddings: Downloading the embeddings is proving challenging.
My first attempt caused Sherwood to crash. A subsequent attempt was
successful, but downloading for multiple landscapes will be
storage-intensive. My proposed solution is to build a pipeline that does
the following:

\pandocbounded{\includegraphics[keepaspectratio]{images/clipboard-355497867.png}}

To this end, I will need to determine which classification approach is
most effective in terms of accuracy and compute. At the moment a lot of
TESSERA demonstrations use an adapted Random Forest method.

\begin{enumerate}
\def\labelenumi{\arabic{enumi}.}
\tightlist
\item
  Project Matching Input Data: To begin matching projects with their
  spillover buffers this month, I need to compile a complete set of
  project geojsons and associated metadata.
\end{enumerate}

\begin{enumerate}
\def\labelenumi{\arabic{enumi}.}
\setcounter{enumi}{1}
\tightlist
\item
  Landscape Profile Data: I need to finalise the datasets for building
  landscape profiles for each project. I have initial ideas (e.g.,
  night-time lights, gridded population, biomass, road maps, access
  layers) but should begin acquiring these resources.
\end{enumerate}

\begin{enumerate}
\def\labelenumi{\arabic{enumi}.}
\setcounter{enumi}{2}
\tightlist
\item
  Feasibility Report: I am worried that my feasibility report is not
  quite striking the balance between being succinct and covering enough
  of what I intend to do. For now, I've included a qualitative political
  ecology chapter on the ethics of proactive leakage management the sort
  I'm advocating for. However, I would be eager to hear your opinions on
  this idea.
\end{enumerate}

\textbf{Plans}

\begin{enumerate}
\def\labelenumi{\arabic{enumi}.}
\tightlist
\item
  Feasibility Report: Finalise the feasibility report promptly to free
  up focus for the pixel matching work.
\item
  Project Database: Compile a folder and table of project geojsons and
  metadata. I will primarily focus on REDD+ projects due to their
  greater policy equivalence, which facilitates cross-regional
  comparison.
\item
  Data Acquisition: Begin systematically downloading the core datasets
  required for building the landscape profiles. For now hold off on
  TESSERA embeddings as they could be a time sink/a distraction whilst
  there is still low hanging fruit.
\item
  Start writing up a short review table of all the existing local
  leakage studies, their findings, regions, and methods. Clearly be able
  to describe what makes my study useful and unique.
\end{enumerate}

\section{Download This Report}\label{download-this-report}

\begin{itemize}
\tightlist
\item
  \href{251106_week3.pdf}{PDF}
\item
  \href{251106_week3.docx}{Word}
\item
  \href{../progress-reports.html}{Back to All Reports}
\end{itemize}

\begin{center}\rule{0.5\linewidth}{0.5pt}\end{center}

\emph{Second progress report}




\end{document}
